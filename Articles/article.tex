\documentclass[english,12pt]{article}
\usepackage[utf8]{inputenc}  

\usepackage[english]{babel} 
\usepackage[T1]{fontenc}
%\usepackage{fullpage}
\usepackage[]{bbm}
\usepackage{amsfonts}
\usepackage{amsmath}
\usepackage{amssymb}
\usepackage{amsthm}
\usepackage{graphicx}
%\usepackage{pstricks}
\usepackage{float}
\usepackage[autolanguage]{numprint}
%\usepackage[squaren,Gray]{SIunits}
%\usepackage{tikz}
%

\title{Sparse associative memories}
\author{L}
\date{
	}
%

\renewcommand{\le}{\leqslant}
\renewcommand{\epsilon}{\varepsilon}
\renewcommand{\Re}{\textrm{Re}}
\renewcommand{\ge}{\geqslant}
%\newcommand{\sgn}{\textrm{le signe de }}
\newcommand{\E}{\mathcal{E}}
\newcommand{\V}{\vect{\mathcal{V}}}
\newcommand{\R}{\mathbb{R}}
\newcommand{\Z}{\mathbb{Z}}
\newcommand{\z}[1]{\Z/#1\Z}
\renewcommand{\P}{\mathbb{P}}
\newcommand{\Q}{\mathbb{Q}}
\newcommand{\N}{\mathbb{N}}
\newcommand{\C}{\mathbb{C}}
\newcommand{\U}{\mathbb{U}}
\newcommand{\f}{\widehat{f}}
\newcommand{\e}[1]{e^{-2 \pi i #1}}
\newcommand{\ex}[1]{e^{2 \pi i #1}}
\renewcommand{\L}[1]{L^{#1} (\R )}
\newcommand{\leg}[2]{\left(\dfrac{#1}{#2}\right)}

\newcommand{\vect}[1]{\overrightarrow{#1}}
\newcommand{\disp}[1]{\displaystyle{#1}}

\setcounter{secnumdepth}{3} %pour la numérotation subsubsection
\setcounter{tocdepth}{3} %pour la numérotation dans la table de matières
\renewcommand{\theenumi}{\roman{enumi})}
\renewcommand{\thepart}{\Alph{part}}
%\renewcommand{\thechapter}{\arabic{chapter}}
\renewcommand{\thesection}{\Roman{section}}
\renewcommand{\thesubsection}{\arabic{subsection}}
\renewcommand{\thesubsubsection}{\alph{subsubsection}}

\newcommand{\fonction}[5]{\begin{array}{cllll}
#1 & : & #2 & \longrightarrow & #3 \\
&    & #4 & \longmapsto & #5 \end{array}}
\newcommand{\fonc}[3]{\begin{array}{lllll}
#1: & #2 & \mapsto & \displaystyle{#3}\end{array}}
\newcommand{\app}[3]{
	#1 : #2 \mapsto \displaystyle{#3}}
\newcommand{\fonct}[3]{\begin{array}{ccccc}
#1: & #2 & \longrightarrow & \displaystyle{#3}\end{array}}
\newcommand{\pent}[1]{\lfloor #1 \rfloor}
\newcommand{\ent}[2]{
	%\comp{#3}{\N}{#1\leqslant #3 \leqslant #2}
	[\![#1\,;\,#2]\!]
	}
\newcommand{\entnb}[1]{
	%\comp{#2}{\N}{#1 \leqslant #2}
	[\![#1\,;\,+\infty[\![
	}
\newcommand{\entr}[2]{
	%\comp{#3}{\Z}{#1\leqslant #3 \leqslant #2}
	[\![#1\,;\,#2]\!]
	}
\newcommand{\entrnb}[1]{
	%\comp{#2}{\Z}{#2 \leqslant #1}
	]\!]-\infty\,;\,#1]\!]
	}
	
% Le point-virgule bien espacé
\newcommand{\pv}{\ensuremath{\, ; }}
% Les intervalles
   % fermé - fermé
   \newcommand{\interff}[2]{\ensuremath{\left[ #1 \pv #2 \right]}}
   % fermé - ouvert
   \newcommand{\interfo}[2]{\ensuremath{\left[ #1 \pv #2 \right[}}
   % ouvert - fermé
   \newcommand{\interof}[2]{\ensuremath{\left] #1 \pv #2 \right]}}
   % ouvert - ouvert
   \newcommand{\interoo}[2]{\ensuremath{\left] #1 \pv #2 \right[}}


\theoremstyle{definition}
\newtheorem{theoreme}{Theorem}
\renewcommand{\thetheoreme}{\arabic{section}.\arabic{theoreme}}


\newcommand{\comp}[3]{\left\{
	#1\,;\, #1\in #2 \, /\, #3
	\right\}}
\newcommand{\param}[2]{\left\{
	#1\, /\, #2
	\right\}}


\begin{document}

	\maketitle%
	%\section*{Introduction}
	
	number of neurons per cluster $l$
	
	number of clusters $c$
	
	number of erased clusters $c_e$
	
	number of messages : $m$
	
	number of activities per cluster : $a$
	
	density : $d$

	edge : $(i, j)$
	
	probability for $i$ and $j$ to be active in their respective clusters for one message : $\frac{a}{l}$
	
	probability of the edge	not being added in the network for one message : $1 - \left(\frac{a}{l}\right)^2$
	
			
	\[ d = 1 - \left( 1 - \left(\frac{a}{l}\right)^2 \right)^m \]
	
	gap with density in simulations : under $1$ percent
	
	probability for a vertex $v$ to achieve the maximum score, that is to mean $a(c-c_e)$ : $d^	{a(c-c_e)}$
	
	probability for a vertex $v$	not to achieve the maximum : $1 - d^	{a(c-c_e)}$
	
	probability for none of the vertices of one erased cluster, excepting the correct ones, to achieve the maximum : $\left(1 - d^	{a(c-c_e)}\right)^{l-a}$
	
	probability for none of the vertices in any erased cluster, excepting the correct ones, to achieve the maximum in their respective cluster : $\left(1 - d^	{a(c-c_e)}\right)^{c_e(l-a)}$
	
	Whence error rate is : \[P_{err} = 1 -	\left(1 - d^	{a(c-c_e)}\right)^{c_e(l-a)} \]
	
	The simulations agree with the analytical result.
	

	For one iteration (for erasures, not for errors), winner takes all is the same as $a$-winners take all.
	
	For multiple iterations $a$-winner take all is an improvement.
	
	
	"$a$-Winners take all" better than "winner takes all" with errors instead of erasures.
	
	For errors : only better than only 1 activity per cluster if multiple iterations of the $a$-winners take all rule.
	For one iteration, it is less efficient.
		
	\newpage
	Spéculations en Français :
	
	Peut-être un intérêt pour cluter based associative memories build from unreliable storage : si une arête porte moins d'info, on peut peut-être en supprimer plus (mais rajout ?!)
	
	Marche un petit peu, mais pas forcément significatif : à tester plus en détail.
	
	\nocite{*}
	%\nocite{*}
	%\bibliographystyle{alpha-fr} % Le style est mis entre crochets.
         \bibliography{bibli}
\end{document}
